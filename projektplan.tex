\documentclass[10pt]{article}
\usepackage[utf8]{inputenc}
\usepackage[swedish]{babel}
\usepackage{pgfgantt}
\usepackage{pgfmath-xfp}
\usepackage{amsmath}

\begin{document}
\definecolor{barblue}{RGB}{153,204,254}
\definecolor{groupblue}{RGB}{51,102,254}
\definecolor{linkred}{RGB}{165,0,33}
\renewcommand\sfdefault{phv}
\renewcommand\mddefault{mc}
\renewcommand\bfdefault{bc}
\setganttlinklabel{s-s}{START-TO-START}
\setganttlinklabel{f-s}{FINISH-TO-START}
\setganttlinklabel{f-f}{FINISH-TO-FINISH}
\sffamily

\section*{\textbf{Projektplan Grupp 11}}

\subsection*{Projektets syfte}
Uppgiften är att utveckla en kamera som styrs av externa ljudkällor. Vid t.ex. ett inbrott ska mikrofoner länkade till kameran kunna detektera ljudet av
ex. en krossad glasruta och trigga kameran till att filma ljudkällan.
\subsection*{Genomförande}
Projektet har 2 tillvägagångssätt, A och B. Plan A går ut på att placera 3 mikrofoner runt kameradonet för att kunna triangulera ljudkällan.
Plan B är att placera ut mikrofoner i "POI" (Point-of-Interest) och lagra dessa positioner i centraldatorn med hjälp av en kalibrering vid setup.
Vilket av dessa tillvägagångssätt som är bäst bestämms efter gruppen har lagt sin första beställning och fått tillgång till komponenterna som kommer användas
för projektet, och testat dess egenskaper (ekon, tid-mot-frekvensupplösning mm). I båda fallen kommer kameran att stryas via en centraldator (Raspberry Pi 3)
som enligt AXIS specifikation ska skicka HTTP-requests till kamerans API, VAPIX. HTTP-klienten kommer ta emot triggers från en mikrokontroller kopplad till
mikrofonerna.
\pagebreak
\subsection*{Delmål}
Nedan är ett gantt-schema som visar projektets delmål och hur de är kopplade till varandra. Målet är att projektet ska vara helt färdigt den 5:e maj, vilket är
12 veckor från att projektets första beställning läggs.

\hspace{-3cm}
\begin{ganttchart}[
    canvas/.append style={fill=none, draw=black!5, line width=.75pt},
    hgrid style/.style={draw=black!5, line width=.75pt},
    vgrid={*1{draw=black!5, line width=.75pt}},
    today=0,
    today rule/.style={
      draw=black!64,
      dash pattern=on 3.5pt off 4.5pt,
      line width=1.5pt,
    },
    today label font=\small\bfseries,
    title/.style={draw=none, fill=none},
    title label font=\bfseries\footnotesize,
    title label node/.append style={below=7pt},
    include title in canvas=false,
    bar label font=\mdseries\small\color{black!70},
    bar label node/.append style={left=2cm},
    bar/.append style={draw=none, fill=black!63},
    bar incomplete/.append style={fill=barblue},
    bar progress label font=\mdseries\footnotesize\color{black!70},
    group incomplete/.append style={fill=groupblue},
    group left shift=0,
    group right shift=0,
    group height=.5,
    group peaks tip position=0,
    group label node/.append style={left=.6cm},
    group progress label font=\bfseries\small,
    link/.style={-latex, line width=1.5pt, linkred},
    link label font=\scriptsize\bfseries,
    link label node/.append style={below left=-2pt and 0pt}
  ]{1}{12}
  \gantttitle[
    title label node/.append style={below left=7pt and -3pt}
  ]{Vecka:\quad1}{1}
  \gantttitlelist{2,...,12}{1} \\
  \ganttgroup[progress=0]{WBS 1: Hårdvara}{1}{12} \\
  \ganttbar[
    progress=0,
    name=WBS1A
  ]{\textbf{WBS 1.1} Första komponentbeställning}{1}{2} \\
  \ganttbar[
    progress=0,
    name=WBS1B
  ]{\textbf{WBS 1.2} Test av komponenter. Val av plan A / B}{2}{3} \\
  \ganttbar[
    progress=0,
    name=WBS1C
  ]{\textbf{WBS 1.3} Tentaplugg/Tentaperiod}{4}{5} \\
  \ganttbar[
    progress=0,
    name=WBS1D
  ]{\textbf{WBS 1.4} Hårdvarudesign: Mikrofoner och microkontroller}{6}{7} \\
  \ganttbar[
    progress=0,
    name=WBS1E
  ]{\textbf{WBS 1.5} Omtentaplugg/Omtentaperiod}{8}{10} \\

  \ganttbar[
    progress=0,
    name=WBS1F
  ]{\textbf{WBS 1.6} Kommunicera mellan MCU och centraldator}{11}{12} \\ [grid]

  \ganttgroup[progress=0]{WBS 2 Mjukvara}{6}{12} \\
  \ganttbar[progress=0]{\textbf{WBS 2.1} Utveckling av HTTP-klient}{6}{11} \\
  \ganttbar[progress=0]{\textbf{WBS 2.2} Kommunicera mellan MCU och centraldator}{11}{12} \\
\end{ganttchart}


\subsection*{Kommunikation}
Kommunikation med handledare kommer föras via mail och möte på AXIS kontor varannan fredag. Intern kommunikation i gruppen kommer hållas via 'messenger', 
eftersom det visat sig fungera under tidigare planering.
\end{document}